\begin{lstlisting}[basicstyle=\ttfamily\footnotesize, frame=single]

// g++ -O3 -std=c++11 find.cpp && ./a.out
#include <iostream>
#include <fstream>
#include <vector>
#include <string>
#include <queue>
#include <cmath>
#include <chrono>
using namespace std;

typedef struct{
  int h;
  int w;
}Point;

typedef struct{
  int template_num;
  int diff;
  Point c_dist;
  double scale;
  int rot;
}Output;

class Image{
public:
  const int MAX = 10000;
  int W, H, W_trim, H_trim;
  double S, S_trim;
  Point upleft;
  vector<vector<int>> data;
  vector<vector<bool>> visited;

  // 画像の読み込み ファイルが存在しない場合falseを返す
  bool readdata(string filename){
    string str;
    ifstream fin(filename);
    if(fin){
      fin >> str; fin.ignore();
      if(str != "P2") cout << "file format error" << endl;
      getline(fin, str); // コメント読み捨て
      fin >> W >> H; 
      data.resize(H);
      for(int i=0; i<H; i++){
        data[i].resize(W);
      }
      fin >> str; //255
      int h=0, w=0;
      while(fin >> str){
        data[h][w] = stoi(str);
        w  = (w+1) % W;
        if(w == 0) h++; 
      }
      makevisited();
      return true;
    }
    else return false;
  }

  // 未踏はfalse, 訪問済みtrue
  // 0が入っている場所はtrue, ついでに総面積S(=画素値の合計)も算出
  void makevisited(){
    visited.resize(H);
    for(int i=0; i<H; i++){
      visited[i].assign(W,false);
    }
    S = 0;
    for(int h=0; h<H; h++){
      for(int w=0; w<W; w++){
        if(data[h][w] == 0){
          visited[h][w] = true;
        }
        else{
          S += data[h][w];
        }
      }
    }
  }

  // 内接する四角の左上端と幅,オブジェクトの画素値の合計を計算
  void trimming(int h, int w){
    queue<Point> que;
    int dp[3] = {-1,0,1};
    que.push({h,w});
    visited[h][w] = true; 
    S_trim = data[h][w];
    int max_h=0, min_h=MAX, max_w=0, min_w=MAX;
    while(que.size()){
      Point p = que.front(); que.pop();
      max_h = max(max_h, p.h);
      min_h = min(min_h, p.h);
      max_w = max(max_w, p.w);
      min_w = min(min_w, p.w);
      for(int i=0; i<3; i++){
        for(int j=0; j<3; j++){
          Point np = {p.h + dp[i], p.w + dp[j]};
          if(np.h>=0 && np.h<H && np.w>=0 && np.w<W){
            if(! visited[np.h][np.w]){
              que.push(np);
              S_trim += data[np.h][np.w];
              visited[np.h][np.w] = true;
            }
          }
        }
      }
    }
    upleft.h = min_h;
    upleft.w = min_w;
    H_trim = max_h - min_h;
    W_trim = max_w - min_w;
  }
};

// 入力画像をscale倍しrot度回転させた画像を返す
Image balance(Image templates, double scale, int rot){
    double theta = rot*M_PI/180;
    Image temp;
    temp.H = templates.H * scale;
    temp.W = templates.W * scale;
    // 28*28のオリジナルtemplate画像から画素を持ってくるので拡大倍率によらず中心は常に(14,14)
    Point c = {templates.H/2, templates.W/2};

    temp.data.resize(temp.H);
    for(int i=0; i<temp.H; i++)
      temp.data[i].resize(temp.W);

    //画像中心を中心として回転
    for(int h=0; h<temp.H; h++){
      for(int w=0; w<temp.W; w++){
        double h_s = h/scale;
        double w_s = w/scale;
        int h_r = cos(-theta)*(h_s-c.h) - sin(-theta)*(w_s-c.w) + c.h;
        int w_r = sin(-theta)*(h_s-c.h) + cos(-theta)*(w_s-c.w) + c.w;
        if(h_r >= templates.H) h_r = templates.H-1;
        if(w_r >= templates.W) w_r = templates.W-1;
        if(h_r < 0) h_r = 0;
        if(w_r < 0) w_r = 0;
        temp.data[h][w] = templates.data[h_r][w_r];
      }
    }
    temp.makevisited();
    return temp;
}

int main(){
  // 時間計測開始


  Image target;
  target.readdata("images/images4/image.pgm");
  
  // template画像がある限り読み込み
  vector<Image> templates;
  for(int i=0; 1;i++){
    templates.resize(i+1);
    if(! templates[i].readdata("images/images4/template"+to_string(i+1)+".pgm")){
      templates.resize(i);
      break;
    }
  }
  auto start = std::chrono::system_clock::now();
  // targetの走査 オブジェクトを見つけるごとに最適なtemplateを探す
  for(int h=0; h<target.H; h++){
    for(int w=0; w<target.W; w++){
      if(! target.visited[h][w]){
        target.trimming(h,w);
        Output ans;
        ans.diff = 2147483647;
        Output now;
        for(int i=0; i<templates.size(); i++){
          now.template_num = i;
          // templateの倍率を画素値の和の比から求め,角度は-90~90まで総当たり
          now.scale = sqrt((double)target.S_trim / (double)templates[i].S);
          for(int r=0; r<180; r++){
            now.rot = r-90;
            Image temp = balance(templates[i], now.scale, now.rot);
            // templateは1度trimmingできればよい 縮小により離れた点を読むことを防ぐ
            bool trimmed = false;
            for(int h_=0; h_<temp.H; h_++){
              for(int w_=0; w_<temp.W; w_++){
                if(! temp.visited[h_][w_] and ! trimmed){
                  temp.trimming(h_, w_);
                  now.c_dist = {temp.H/2-temp.upleft.h, temp.W/2-temp.upleft.w};
                  trimmed = true;
                }
              }
            }
            // 左上の点を合わせて比較   範囲は小さい方に合わせる
            int diff = 0;
            Point range = {min(target.H_trim,temp.H_trim), min(target.W_trim, temp.W_trim)};
            for(int dh=0; dh<range.h; dh++){
              for(int dw=0; dw<range.w; dw++){
                int temp_pixel = temp.data[temp.upleft.h+dh][temp.upleft.w+dw];
                int target_pixel = target.data[target.upleft.h+dh][target.upleft.w+dw];
                // 1画素あたりの画素値の差の2乗を求める
                diff += pow((temp_pixel - target_pixel), 2) / (range.h * range.w) ;
              }
            }
            // diffが小さい方を答えとする
            if(diff < ans.diff){  
              now.diff = diff;  
              ans = now;
            }
          }
        }
        int center_h = target.upleft.h + (ans.c_dist.h);
        int center_w = target.upleft.w + (ans.c_dist.w);
        cout << "template" << ans.template_num+1 << "  " << center_w << "  " << center_h 
             << " " << ans.rot << "  " << ans.scale << endl;
        ofstream ofs("result.csv",ios::app);
        ofs << "template" << ans.template_num+1 << "," << center_w << "," << center_h 
             << "," << ans.rot << "," << ans.scale << endl;
      }
    }
  }

  auto end = std::chrono::system_clock::now();     
  auto dur = end - start;       
  auto msec = std::chrono::duration_cast<std::chrono::milliseconds>(dur).count();
  std::cout << msec << " milli sec \n";
  return 0;
}


\end{lstlisting}